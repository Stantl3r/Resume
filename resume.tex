%Resume template found on overleaf.com

\documentclass{resume} 

\usepackage[left=0.75in,top=0.6in,right=0.75in,bottom=0.6in]{geometry}
\newcommand{\tab}[1]{\hspace{.2667\textwidth}\rlap{#1}}
\newcommand{\itab}[1]{\hspace{0em}\rlap{#1}}
\name{Stanley Tran} 
\address{(408)~603-0839 \\ stanleyltran@gmail.com} 

\begin{document}


\begin{rSection}{Education}

{\bf University of California, Santa Cruz} \hfill {GPA: 4.00} 
\\ B.S., Computer Science \hfill {Expected Graduation: 06/2021}

\end{rSection}


\begin{rSection}{Experience}

\begin{rSubsection}{Tech Mentor}{06/2018 - 09/2018}{West Valley Library}{}
\item Provided assistance to customers of all ages regarding computers, phones, tablets, etc.
\item Fixed software issues with computers, such as internet and driver problems, and demonstrated how to use certain applications
\item Assisted customers with tasks, including resume formatting and spreadsheet manipulation
\end{rSubsection}

\begin{rSubsection}{Fundraising Chair}{09/2018 - Present}{Society of Asian Scientists and Engineers}{}
\item Increase the number of events and funding througout the school year by executing fundraisers
\item Organize the layout of the SASE West Regional Conference through managing the logistics committee
\item Plan successful funding events by preparing ahead of time and completing the proper paperwork
\end{rSubsection}

\end{rSection}


\begin{rSection}{Personal Projects}

\begin{rSubsection}{Bothoven}{07/2018}{}{}
\item Created a Discord bot that allows the user to input a song and outputs a list of song predictions based on the similarity of emotion
\item Implemented natural language understanding and Spotify API to analyze the emotion of a song's lyrics
\item Calculated the error by comparing each song's emotion in order to determine how similar they are to one another
\end{rSubsection}

\begin{rSubsection}{Botty Flay}{08/2018}{}{}
\item Developed a script that has the user input a list of ingredients and then outputs a list of recommended recipes that uses those ingredients
\item Utilized web scraping to obtain recipe information to check whether the ingredients matched the user's input
\item Incorporated the ability for the user to sort the recipes by difficulty level
\end{rSubsection}

\begin{rSubsection}{Doggit}{08/2018}{}{}
\item Constructed an image classifier by creating a convolutional neural network in order to predict which dog breed is in the picture
\item Used TFLearn to build the convolutional neural network and trained the model using approximately 14,000 images of different dog breeds
\item Included Reddit API to analyze image submissions and respond to comments when the bot is invoked
\item The bot currently supports 14 different breeds with an accuracy of 71\%
\end{rSubsection}

\end{rSection}


\begin{rSection}{Technical Skills}

\begin{tabular}{ @{} >{\bfseries}l @{\hspace{6ex}} l }
Languages &  Python, Java, JavaScript, C, HTML, CSS \\
Technologies \& Editors & Git, Unix, IntelliJ, Eclipse, Mac OS X, Windows \\
\end{tabular}

\end{rSection}


\begin{rSection}{Relevant Courses}
\itab{\textbf{Core Courses}} \tab{}  \tab{\textbf{Other Courses}}
\\ \itab{Introduction to Programming} \tab{}  \tab{Vector Calculus}
\\ \itab{Computer Systems \& Assembly Language} \tab{}  \tab{Linear Algebra} 
\\ \itab{Applied Discrete Mathematics} 
\\ \itab{Introduction to Data Structures} 
\end{rSection}


\end{document}
